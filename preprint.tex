\documentclass[draaft,preprint,12pt,2023]{elsarticle}

\graphicspath{{images/}}

\usepackage{amssymb}
\usepackage{amsmath}
\usepackage{makecell}
\usepackage{subfig}
\usepackage{float}
\usepackage{siunitx}

\usepackage{
	siunitx,
	tabularx,
	array,
	booktabs
	}
\sisetup{
	round-mode = places,
	round-precision = 0,
	per-mode = symbol,
	bracket-unit-denominator = false
}
 
\PassOptionsToPackage{hyphens}{url}\usepackage{hyperref}

\journal{Energy and Built Environment}

\begin{document}

\begin{frontmatter}

\title{Thermoenergetic performance of a building with phase change materials in future climates}

\author[1]{Lorenzo Olivo Filippini\corref{cor1}}
\ead{filippini.lor@gmail.com}

\author[1]{Letícia Jenisch Rodrigues} 
\ead{leticia.jenisch@gmail.com}

\author[2]{Marcelo Schramm}
\ead{schrammmarcelo@gmail.com}

\cortext[cor1]{Corresponding author}

\affiliation[1]{organization={School of Engineering, Federal University of Rio Grande do Sul},%Department and Organization
            addressline={Argentina Square}, 
            city={Porto Alegre},
            postcode={90040-020}, 
            state={Rio Grande do Sul},
            country={Brazil}}

\affiliation[2]{organization={Center of Engineering, Federal University of Pelotas},%Department and Organization
        addressline={Domingos Rodrigues Square}, 
        city={Pelotas},
        postcode={96010-450}, 
        state={Rio Grande do Sul},
        country={Brazil}}

\begin{abstract} 
This paper investigates the thermal and energy performance of a building with different envelope compositions and its responses to future climate scenarios considering climate change projections. Through computer simulations using EnergyPlus, the building envelope is altered by applying resistive thermal insulation, phase change material, or no insulating material at all, and simulated in bioclimatic zones 1 and 8, according to the Brazilian Association of Technical Standards (ABNT). The building, of the light framing type, is built according to American Society of Heating, Refrigeration and Air-Conditioning Engineers (ASHRAE) guidelines. The phase change material used, SP24E, has operating temperatures consistent with the setpoints of the modeled HVAC system. Employing the Climate Change World Weather File Generator (CCWorldWeatherGen) tool, climate change behaviors for the representative years 2050 and 2080 are incorporated into the climate files through the statistical downscaling technique \textit{morphing}, considering the A2 emissions scenario (medium-high) of the Intergovernmental Panel on Climate Change (IPCC). The results show that in mild climates, such as in Curitiba, the phase change material proves to be more efficient than the resistive thermal insulator under climate change scenarios, reducing the annual Energy Use Intensity (EUI) projected for 2050 by 8.2\% and for 2080, by 10\%. For Rio de Janeiro, the envelope with phase change material in the climate projections results in higher EUI, registering 12.1\% in 2050 and 20.7\% more in 2080 than the envelope with resistive thermal insulator.
\end{abstract}

%%Graphical abstract
\begin{graphicalabstract}
%\includegraphics{grabs}
\end{graphicalabstract}

%%Research highlights
\begin{highlights}
\item Comparison between phase change materials and ASHRAE 90.1 default thermal insulation in future climates
\item Assessment of building performance with different envelope configurations in climate change projections
\item Behavior of phase change materials in changing and hotter climate conditions 
\item Phase change materials can be effective in today's climate, but might not work as intended in future warmer climates
\end{highlights}

\begin{keyword}
Phase change materials \sep Thermal insulation \sep Building Envelope \sep Climate change projections \sep Building energy performance \sep Future climates
\end{keyword}

\end{frontmatter}

%% \linenumbers

%% main text
\section{Introduction}
The building and civil construction sectors were responsible for around 30\% of final energy use and 27\% of CO2 emissions worldwide in 2021 \cite{ieabuildings}. In the Brazilian context, for the same year, buildings comprised around 51\% of the country's final electricity consumption \cite{epe}. Global consumption with air-conditioning and ventilation in 2021 corresponded to roughly 16\% of the final use of electricity in buildings, equivalent to 10\% of the total electricity use in the world in that same year \cite{ieacooling}. 

Given the worsening of climate change and extreme temperature events, the demand for air-conditioning systems only tends to grow, and, in the next three decades, the use of air-conditioning systems will be one of the main drivers of global electricity demand \cite{ieafuture}. For the building sector to comply with and contribute to the net zero scenarios planned for 2030 onward, significant efforts must be implemented throughout the production and operational chain. The existential ``paradox'' of air-conditioning systems, which simultaneously guarantee comfortable climates and indoor conditions but reject the heat into an increasingly hotter atmosphere, denotes the imperative of applying energy efficiency measures and technological harmony in buildings and their systems. 

In response to this, the search for passive and active alternatives that ensure the thermal resilience of the built environment and thermal comfort over the long term, taking into account future climate changes, must be applied since the building's conception. Accordingly, the envelope contributes to around 60\% of the energy cost during the life cycle of a building \cite{yuksek2017energy}. The appropriate project incorporating high-performance envelope technology is the most effective strategy to mitigate the thermal needs of buildings and ensure the comfort of occupants \cite{ieaenvelope}. Envelope incorporation of thermal energy storage through latent heat using Phase Change Materials (PCM) is considered an advanced and promising technology \cite{liu2018review}.

With this stated, this paper aims to evaluate the thermal and energetic performance of a building with different proposed envelopes, using phase change materials, considering future climate change scenarios of distinct climates in Brazil. 

The building object of study represents the typology of a medium office building and has an energy model characterized by a single geometry. However, it has construction and Heating, Ventilation, and Air-Conditioning (HVAC) systems in line with the American Society of Heating, Refrigeration and Air-Conditioning Engineers (ASHRAE) standards. The models suitable for the different climates in Brazil are simulated using the EnergyPlus software, version 9.2 \cite{energyplus2019}, in the cities of Curitiba and Rio de Janeiro representing, respectively, bioclimatic zones 1 and 8, according to the Brazilian Association of Technical Standards (ABNT).

By default, the buildings rely on purely resistive thermal insulation to adapt the thermal parameters of the external walls and roof established by ASHRAE. The work aims to compare the performances of the reference building without any insulating material with the designed resistive thermal insulator and the same building in which a PCM geometrically replaces the insulator. The same PCM is applied in both climatic regions, and its operating temperature matches the setpoints of the existing air-conditioning system. Projections for climate change are incorporated into the simulations through the climate files, using the statistical \textit{morphing} process and data from the A2 (medium-high) emissions scenario, defined and made available by the Intergovernmental Panel on Climate Change (IPCC). The process is carried out using the CCWorldWeatherGen tool, version 1.9 \cite{jentsch2013transforming} from the University of Southampton \cite{ccworldweathergen}. Two periods of climate change are constructed for the reference years of 2050 and 2080, taking the TMYx-type climate file as the basis for changes.

To understand the climatic consequences and the responses of the cited envelopes, energy consumption with the HVAC systems and the net heat balance across the surfaces of the north and south façades and roof are studied on a monthly scale. Mean radiant temperatures, operative temperatures, and heat fluxes through the internal faces of envelope surfaces in characteristic thermal zones are evaluated on an hourly scale, during a typical week, to assess the reasons for the different consumption and net heat balance profiles. Emphasizing the PCM application, the behavior of the material's physical state applied to each surface is studied to understand the reaction of the selected material to future climates and its impact on other metrics.
According to a study from 2022, \cite{d2022will}, for the climate projected for the year 2060 in several European cities, an increase between 99 and 380\% in the cooling thermal load is estimated, while the opposite behavior can be observed for the heating load, which has an estimated reduction between 38 and 57\%, depending on the typology and location. The results demonstrate that zero-energy buildings with lower insulation levels are less sensitive to climate change. Accordingly, buildings in hot climates in southern Italy, within the legal limits for the envelope, will suffer from overheating in the future (2080 projection), leading to disproportionate energy consumption with air-conditioning \cite{baglivo2022long}. Regulatory limits for envelopes have shown trends toward increasing thermal insulation, characterized by low thermal transmittance. 

Among the solutions that can be implemented in the envelope, it is possible to mention the increase in the thermal mass of the surfaces -- improving the energy storage and thermal lag -- and the increase in the thermal resistance of the layers of these surfaces -- reducing heat transfer by conduction \cite{zilberberg2021impact}. Although common construction materials can store thermal energy spontaneously as sensible heat, the use of latent heat is considerably more interesting from the point of view of the energy density of the process \cite{rathore2022thermal}. Bulk thermal insulators are capable of significantly reducing heat flow, on the other hand, they have little influence on energy storage \cite{wang2022critical}. In this context, the utilization of phase change materials to store thermal energy through latent heat presents a compelling option for mitigating thermal loads associated with heating and cooling, providing an innovative alternative to traditional envelope construction.

Phase change materials have a dynamic behavior, as they can change their physical state at a specific and almost constant operating temperature during the process. Therefore, the passive or active use of PCM, linked to other strategies, can favor their application scenarios. A study from 2021 \cite{olivo2021impact} evaluated the impact of applying PCM on the thermal comfort of a public school with a standardized project in Brazil's bioclimatic regions 1 and 8 \cite{abnt15220}, also with Curitiba and Rio de Janeiro as representative cities. The buildings did not have mechanical HVAC systems and used natural ventilation to maintain thermal comfort, assessed based on 2017's ASHRAE Standard 55. The authors report that the PCM with a melting temperature of \SI{21}{\celsius} obtained significant results in improving thermal comfort in both cities when applied to the external walls and with a thickness of \SI{5}{\centi\metre}. For the city of Rio de Janeiro, the PCM with a melting temperature of \SI{27}{\celsius} also demonstrated a reduction in thermal discomfort hours when applied to the roof, also \SI{5}{\centi\metre} thick.

\section{Literature Review}
The envelope is the central building system that reacts to local weather conditions. The trends towards intensifying climate change and rising temperatures at a global level, significant challenges for humanity in the following years, will directly impact the performance of buildings and energy consumption throughout their existence. Remodeling the building envelope proves to be an essential strategy from the point of view of several aspects: energy savings, improvement of indoor microclimates, reduction of pollutants in manufacturing, and technical and economic feasibility \cite{yuan2022thermal}. Constructive elements with high thermal performance, which comprise the envelope surfaces, guarantee reduced energy consumption with air-conditioning and improved thermal comfort ite{fang2014effect}.

According to a study from 2022 \cite{d2022will}, for the climate projected for the year 2060 in several European cities, an increase between 99 and 380\% in the cooling thermal load is estimated, while the opposite behavior can be observed for the heating load, which has an estimated reduction between 38 and 57\%, depending on the typology and location. The results demonstrate that zero-energy buildings with lower insulation levels are less sensitive to climate change. Accordingly, buildings in hot climates in southern Italy, within the legal limits for the envelope, will suffer from overheating in the future (2080 projection), leading to disproportionate energy consumption with air-conditioning \cite{baglivo2022long}. Regulatory limits for envelopes have shown trends toward increasing thermal insulation, characterized by low thermal transmittance. 

Among the solutions that can be implemented in the envelope, it is possible to mention the increase in the thermal mass of the surfaces -- improving the energy storage and thermal lag -- and the increase in the thermal resistance of the layers of these surfaces -- reducing heat transfer by conduction \cite{zilberberg2021impact}. Although common construction materials can store thermal energy spontaneously as sensible heat, the use of latent heat is considerably more interesting from the point of view of the energy density of the process \cite{rathore2022thermal}. Bulk thermal insulators are capable of significantly reducing heat flow, on the other hand, they have little influence on energy storage \cite{wang2022critical}. In this context, the utilization of phase change materials to store thermal energy through latent heat presents a compelling option for mitigating thermal loads associated with heating and cooling, providing an innovative alternative to traditional envelope construction.

Phase change materials have a dynamic behavior, as they can change their physical state at a specific and almost constant operating temperature during the process. Therefore, the passive or active use of PCM, linked to other strategies, can favor their application scenarios. A study from 2021 \cite{olivo2021impact} evaluated the impact of applying PCM on the thermal comfort of a public school with a standardized project in Brazil's bioclimatic regions 1 and 8 \cite{abnt15220}, also with Curitiba and Rio de Janeiro as representative cities. The buildings did not have mechanical HVAC systems and used natural ventilation to maintain thermal comfort, assessed based on 2017's ASHRAE Standard 55. The authors report that the PCM with a melting temperature of \SI{21}{\celsius} obtained significant results in improving thermal comfort in both cities when applied to the external walls and with a thickness of \SI{5}{\centi\metre}. For the city of Rio de Janeiro, the PCM with a melting temperature of \SI{27}{\celsius} also demonstrated a reduction in thermal discomfort hours when applied to the roof, also \SI{5}{\centi\metre} thick.

Following this train of thought, phase change materials' reaction is sensitive to climate conditions, although a particular range of operating temperatures can show interesting results in different climates. The authors from a 2022 study \cite{kalbasi2022buildings} applied distinct PCM to buildings with envelopes appropriate to each climate classification, complying with ASHRAE Standard 90.1. With the application of PCM, the period of thermal comfort varied from 54 to 82\% among the 10 climates analyzed, in which the performance of PCM with melting points of \SI{22}{\celsius} (suitable for 6 regions) and \SI{23}{\celsius} (suitable for 8 regions) can be highlighted. The authors also mention that applying the material close to the interior of the building translates into more extended periods of thermal comfort than scenarios with other implementation positions. Correspondingly, a 2021 work \cite{carlucci2021phase} concluded that the application of PCM significantly reduces the thermal load of cooling environments. In the study carried out, a plasterboard with microencapsulated PCM application and a melting point of \SI{23}{\celsius} -- Alba Balance produced by Saint Gobain -- is applied to buildings in different European climates. 

Regarding resistive thermal insulators applied in the Brazilian national context, the maximum limits of thermal transmittance of walls given by ASHRAE Standard 90.1 from 2013 are not suitable for hot regions such as Brazil's \cite{melo2015thermal}, in which energy consumption with air-conditioning is dominated by cooling. Thermal insulation can increase the annual thermal load, reducing spontaneous heat exchange from the internal to the outdoor environment at night. Collaborating with such an outcome, a study from 2022 \cite{kalbasi2022one} attests that the use of PCM on walls produces better results during summer than thermal insulation.

The mentioned results and reactions define the guiding hypothesis of the present work: in the climate change scenario characterized by higher temperatures, using PCM in the envelope generates better thermal and energy performance results than bulk or purely resistive thermal insulators.

\section{Theoretical Foundation}
Phase change materials are materials capable of changing their physical state at a convenient temperature, being applied in various industry sectors (food, energy, aerospace, automotive, electronics, among others) for storing thermal energy through latent heat, the process also known as Latent Heat Thermal Energy Storage (LHTES). 
\subsection{Selection criteria and application methods of phase change materials}
The use of phase change materials depends mainly on the desired operating temperature since, during the physical state transition process, the material presents slight temperature variation, transforming the transferred energy into enthalpy variation. This way, the material selection starts from determining this temperature \cite{suresh2022phase}. Other characteristics of PCM must be observed to extract the most significant potential from its implementation, for example, its thermodynamic, chemical, kinetic, and economic properties \cite{rathore2022thermal,liu2018review}, summarized below: 
\begin{itemize}
    \item Thermodynamics: adequate phase change temperature, low phase segregation, high latent heat of fusion per unit volume, high specific heat, high thermal conductivity, and low volume variation between physical states.
    \item Chemicals: non-toxic, low flammability, non-explosive, low corrosivity of construction materials, long-term chemical stability, complete melting and solidification cycles.
    \item Kinetics: high nucleation rate, absence of subcooling, high crystalline growth rate.
    \item Economical: low cost, commercial availability.
    \item Others: low environmental impact, non-polluting, recyclable, compatibility with the capsule material.
\end{itemize}

The application of PCM in civil construction can be carried out through different methods \cite{liu2018review}: direct incorporation, immersion, encapsulation, and shape stabilization. Among these, encapsulation is the most promising method for buildings. The form of encapsulation can also be divided according to the scale of the capsule: nanoencapsulated (1 to \SI{1000}{\nano\metre}), microencapsulated (1 to \SI{1000}{\micro\metre}), and macroencapsulated ($>$\SI{1000}{\micro\metre}) \cite{paroutoglou2019pcm}. Techniques involving low capsule scales ($<$\SI{1000}{\micro\metre}) incorporate the materials directly into the construction elements, changing the thermal properties of the final system. Macroencapsulation is the most promising application in the buildings' context \cite{liu2018review}. 

\subsection{Classification of phase change materials}
The primary classification of phase change materials depends on the physical states in which the material will operate: solid-solid, solid-gas, solid-liquid, and liquid-gas \cite{paroutoglou2019pcm}. Solid-liquid PCM is intended in civil construction due to the extensive range of operating temperatures of materials with such behavior. Figure \ref{fig:operating temps} shows melting temperature and specific enthalpy regions of different solid-liquid PCM compositions.  
\begin{figure}[H]
    \centering
    \includegraphics[width=0.70\textwidth]{pcm working ranges}
    \caption{Operating enthalpy and melting temperature ranges of solid-liquid PCM}
    \label{fig:operating temps}
\end{figure}
Within this macro group of solid-liquid PCM, there are also other classifications based on the origin and chemical composition of the material. Depending on the classification, these can have different properties, translating into advantages or disadvantages. A brief description of solid-liquid PCM compositions is presented as follows \cite{veerakumar2016phase}: 
\begin{itemize}
    \item Organic: carbon-based compounds, generally classified as paraffinic and non-paraffinic.
    \item Inorganic: generally hydrated or metallic salts.
    \item Eutectics: mixtures of two or more PCM to obtain desired operating temperatures. They are also divided based on the melting temperature (high or low) and also by the components of the mixture, being organic, inorganic, or both.
\end{itemize}
\subsection{Thermal energy storage by latent heat and thermal hysteresis}
Materials can store heat in three ways: sensible heat, latent heat, and chemical reactions. The process that PCM perform, regardless of their classification, is modeled by energy balance as \cite{akeiber2016review}

\begin{equation}
Q = \int_{T_{ini}}^{T_{pc}} m\,c_p(T) \,dT\ +
m\,a_{pc}\,\Delta{h_{pc}} + \int_{T_{pc}}^{T_{fin}} m\,c_p(T) \,dT\
\label{eq 1}
\end{equation}

which denotes the energy stored ($Q$) in the mass of material ($m$) within a range of temperatures $T$, $c_p(T)$ denotes the specific heat of the material as a function of temperature and $\Delta{h_{pc}}$ is the specific enthalpy variation of the material in the process. The sub-indices ``$ini$'' and ``$fin$'' refer to the initial and final states, respectively. The subindex ``$pc$'' refers to the phase change point. At this temperature value, the physical state of the material effectively changes. Ideally, this change occurs without temperature variations in temperature. However, in real materials, there may be a slight variation represented by the coefficient $a_{pc}$. The integrals refer to sensible heat storage, depending on the mass of the material and its specific heat. The phase change is represented by the enthalpy change of the material (right-hand side's central term in Equation \eqref{eq 1}). The thermal energy received or given up during the phase change comes from the PCM enthalpy conversion.

\begin{figure}[H]
    \centering
    \includegraphics[width=0.70\textwidth]{hysteresis curve}
    \caption{Ideal thermal hysteresis curve performed by a phase change material.}
    \label{fig:hysteresis}
\end{figure}

A PCM's cyclic phase change process characterizes its thermal hysteresis phenomenon (Figure \ref{fig:hysteresis}). The occurrence in empirical applications and numerical simulations of phase change materials highlight the phenomenon's importance for thermal performance and durability \cite{zastawna2020novel, al2019application}. Equation \eqref{eq 1} applies to both curves (melting or solidification) and the area between them establishes the energy the material stores during phase change. 

\subsection{EnergyPlus simulation}
The EnergyPlus software is one of the few whole building simulation tools capable of modeling phase change materials \cite{al2019application}. According to the software's Engineering Reference \cite{engreference}, heat conduction on composite surfaces uses Conduction Transfer Functions (CTF) algorithms developed by the software's precursors, capable of solving the differential equations related to the conduction process. The state space method applies finite differences to solve the equations using thermal resistances and capacitances concepts. Figure \ref{fig:node system} presents an interpretation of the algorithm for a system with two nodes. 

\begin{figure}[!h]
    \centering
    \includegraphics[width=0.60\textwidth]{system nodes}
    \caption{Construction system with two nodes.}
    \label{fig:node system}
\end{figure}

Note the representations of thermal resistances ($R$) and capacitances ($C$) linked to each node and which make up the state space used in the solution, visible through the equations 

\begin{equation}
\begin{bmatrix}
\frac{dT_1}{dt} \\
\frac{dT_2}{dt} 
\end{bmatrix} = 
\begin{bmatrix}
\frac{-1}{RC}-\frac{hA}{C} && \frac{1}{RC} \\
\frac{1}{RC} && \frac{-1}{RC}-\frac{hA}{C}
\end{bmatrix}
\begin{bmatrix}
T_1\\
T_2
\end{bmatrix} + 
\begin{bmatrix}
\frac{hA}{C} && 0 \\
0 && \frac{hA}{C}
\end{bmatrix}
\begin{bmatrix}
T_o\\
T_i
\end{bmatrix}
\label{eq 2}
\end{equation}
\begin{equation}
\begin{bmatrix}
q^{\prime\prime}_o \\
q^{\prime\prime}_i 
\end{bmatrix} = 
\begin{bmatrix}
0 && -h \\
h && 0
\end{bmatrix}
\begin{bmatrix}
T_1\\
T_2
\end{bmatrix} + 
\begin{bmatrix}
0 && h \\
-h && 0
\end{bmatrix}
\begin{bmatrix}
T_o\\
T_i
\end{bmatrix}
\label{eq 3}
\end{equation}

In Equations \eqref{eq 2} and \eqref{eq 3}: $T$ represents the node temperature, $A$ is the surface area, $h$ is the convective heat transfer coefficient and $q^{\prime\prime}$ identifies the heat flux, where the subindexes ``$o$'' and ``$i$'' refer to the outdoor or indoor. This algorithm is the software's default and can be used to simulate conventional construction materials. 

The algorithms use constant thermal properties for the surface layers and do not produce results for its core. Therefore, given the change in properties during the phase transition, using a more fundamental algorithm capable of incorporating the dynamic behavior of the PCM, is necessary. To this end, based on user configuration, the software implements the one-dimensional finite difference algorithm Conduction Finite Difference (CondFD) \cite{engreference}. Through it, surface layers are discretized dynamically based on the material's thermal properties. Figure \ref{fig:discretization} presents this discretization on a complex surface, illustrating the types of nodes the algorithm considers. \begin{figure}
    \centering
    \includegraphics[width=0.95\textwidth]{discretization}
    \caption{Spatial discretization of the one-dimensional finite difference method.}
    \label{fig:discretization}
\end{figure}

The spatial discretization of each layer of material is carried out by dividing the layer's thickness by a variable that is dependent on the thermal properties of the material and by the time step defined by the user

\begin{equation}
x = \sqrt{c\,\alpha\,\Delta{t}}
\label{eq 4}
\end{equation}

The integer result of this division provides the number of nodes used in the operation. The spacing between the nodes ($\Delta{x}$) is defined by dividing the thickness of the material by the number of nodes. In Equation \eqref{eq 4}, $\alpha$ represents the thermal diffusivity of the material layer, $\Delta{t}$ the time interval, and $c$ is the spatial discretization constant, which basically depicts the inverse of the numerical Fourier Number \cite{engreference} 

\begin{equation}
Fo = \frac{\alpha\,\Delta{t}}{\Delta{x}^2}
\label{eq 5}
\end{equation}

Hence, the spatial discretization constant of the CondFD method is based on the Von Neumann (or Fourier) stability criterion for an explicit solution. This dimensionless value represents the relative effectiveness with which a solid material can conduct and store thermal energy \cite{pakowski2007fundamentals}. In this case, $\Delta{x}$ identifies the spacing between nodes and $\Delta{t}$ the time step for each iteration, as in Equation \eqref{eq 4}. 

The finite difference method is applied to the one-dimensional heat equation \cite{pakowski2007fundamentals}, represented by

\begin{equation}
\rho\,c_p\,\frac{\partial{T}}{\partial{t}}=\frac{\partial}{\partial{x}}
\left({k\frac{\partial{T}}{\partial{x}}}\right) + \Dot{q}
\label{eq 6}
\end{equation}
in which the left-hand side represents the stored energy, the first term on the right-hand side represents the heat flow and $\Dot{q}$ represents the volumetric heat generation rate in the control volume. The expression denotes that, at any point in the medium where conduction occurs, the net rate of heat transfer in the control volume plus the rate of thermal energy generation must be equal to the rate of change of energy stored in the volume. 

In other words, EnergyPlus considers the PCM as a solid medium throughout the simulation, updating the thermal properties of the material nodes as it changes its physical state. Convection phenomena, buoyancy effects, or other liquid phase behaviors are not taken into account. As an alternative for solving the finite difference equation, the software presents the possibility of applying two discretization schemes 

\begin{equation}
\begin{split}
\rho\,c_p\,\Delta{x}\frac{T^{j+1}_{i}-T^{j}_{i}}{\Delta{t}}=
\frac{1}{2}&\left(k_W\frac{T^{j+1}_{i+1}-T^{j+1}_{i}}{\Delta{t}}+k_E\frac{T^{j+1}_{i-1}-T^{j+1}_{i}}{\Delta{t}}\right.+\\
&\,\,\,\,\left.k_W\frac{T^{j}_{i+1}-T^{j}_{i}}{\Delta{t}}+k_E\frac{T^{j}_{i-1}-T^{j}_{i}}{\Delta{t}}\right)\
\end{split}
\label{eq 7}
\end{equation}
\begin{equation}
\rho\,c_p\,\Delta{x}\frac{T^{j+1}_{i}-T^{j}_{i}}{\Delta{t}}=
\frac{1}{2}\left(k_W\frac{T^{j+1}_{i+1}-T^{j+1}_{i}}{\Delta{t}}+k_E\frac{T^{j+1}_{i-1}-T^{j+1}_{i}}{\Delta{t}}\right)
\label{eq 8}
\end{equation}

Equation \eqref{eq 7} identifies Crank Nicholson's semi-implicit second-order scheme and Equation \eqref{eq 8} the fully implicit first-order scheme, both based on the Adams-Moulton solution \cite{engreference}. The subindexes ``$W$'' and ``$E$'' refer, respectively, to the nodes to the left (West) and the right (East) of the modeled node, identified by the subindex ``$i$'', while``$i+1$'' identifies the adjacent node in the direction towards the outside of the building wall and ``$i-1$'' the adjacent node towards the inside of the construction. The indexes ``$j$'' and ``$j+1$'' refer to the temporal discretization where ``$j$'' is the current time step and ``$j+1$'' is the next one. 

Being the solution implicit, the iterative Gauss-Seidel method is used with a sub-relaxation coefficient to increase numerical stability. The iterative process mentioned is the deepest solver of the CondFD method and is applied to all surfaces. The enthalpy value at each of the nodes is calculated in each iteration. EnergyPlus also performs iterations in an external loop considering the internal energy balance in each surface layer to guarantee thermal exchanges arising from long-wave radiation. 

To consider the hysteresis phenomenon, the software requires thermodynamic information about the material in both physical states of the process, being able to construct the curves that will compose the hysteresis. While the enthalpy of the nodes is calculated in each of the iterations, the specific heat value of the material is also updated, depending on the temperatures and physical states of the node at the beginning and end of the time step. Tolerance values for numeric convergence are maintained as the default ones defined by EnergyPlus. 

\subsection{\textit{Morphing} procedure} %revisar
The fundamental \textit{morphing} process changes the original climate variables based on the climate projections of a Global Circulation Model (GCM), but it maintains the meteorological behavior of the base period for modification (baseline). The form of the baseline climate depends on the data projected in the climate change scenario. The developed method uses scenarios that list changes in monthly mean weather variables \cite{belcher2005constructing}. Thus, the baseline climate behavior is calculated separately for each month using 
\begin{equation}
\langle p_0 \rangle _m = \frac{1}{24\,d_m\,N}\sum_{N}\sum_{m}p_0
\label{eq 9}
\end{equation}
where the variable $p_0$ denotes the present-day weather record for each month $m$. The climatological reference of this variable in each month is expressed by $\langle p_0 \rangle _m$, defined as the average of $p_0$ for the month $m$ in all the $N$ years of the sample. The number of days in each month is represented by $d_m$ and the hourly average of the variable by $24$. The $\langle p_0 \rangle _m$ is the reference for direct comparison with the equivalent variable in the climate projections. Depending on the result of this comparison, only one of the operations is applied 
\begin{equation}
p=p_0+\Delta{p_m}
\label{eq 10}
\end{equation}
\begin{equation}
p=\alpha_m\,p_0
\label{eq 11}
\end{equation}
\begin{equation}
p=p_0+\Delta{p_m}+\alpha_m(p_0-\langle p_0 \rangle _m)=\langle p_0 \rangle _m+\Delta{p_m}+(1+\alpha_m)(p_0-\langle p_0 \rangle)_m
\label{eq 12}
\end{equation}
which denote, respectively, the operations of shift, stretch, and the combination of both \cite{belcher2005constructing}. 

The shift operation (Equation \eqref{eq 10}) is applied when, in the climate change scenario, there is an absolute variation in the monthly mean of variable $p_0$. The stretch operation (Equation \eqref{eq 11}) is used when the projected scenario shows partial changes in the mean or monthly variance of the variable or when the variable presents a specific and constant behavior during a period, for example, solar irradiation, which is zero during the night. The combination of the shift and stretch operations (Equation \eqref{eq 12}) is applied when both the mean and variance of the variable present changes.  

In Equations \eqref{eq 10} and \eqref{eq 12}, $\Delta{p_m}$ expresses the absolute variation of the monthly mean of the variable for the month $m$. As a consequence of the shift operation, the monthly variance of the variable is not changed and the calculation of the monthly mean of $\langle p\rangle_m$ becomes 

\begin{equation}
\langle p\rangle_m = \langle p_0\rangle_m + \Delta{p_m}
\label{eq 13}
\end{equation}

In Equations \ref{eq 11} and \ref{eq 12}, the scale factor $\alpha_m$ identifies a non-integer variation of $\langle p\rangle_m$. The stretch operation alters the monthly mean and monthly variance of the variable 
\begin{equation}
\langle p\rangle_m = \alpha_m\,\langle p_0\rangle_m
\label{eq 14}
\end{equation}
\begin{equation}
\langle \sigma^2\rangle_m = \alpha^2_m\,\langle \sigma^2_0\rangle_m
\label{eq 15}
\end{equation}

\section{Methodology and modeling}
To carry out this work, a generic thermoenergetic model is utilized. Given the magnitude of the entire Brazilian territory, representative cities with different climatic behaviors are selected to evaluate the performance of the building and PCM in various application scenarios. Such behaviors are conveyed through weather files of the Typical Meteorological Year (TMY) type, which are modified using the CCWorldWeatherGen tool. As a result, two new files are created to represent future climate change projections: one for 2050 and another for 2080, with TMY as the reference.

The thermoenergetic model is provided and made available by the United States Department of Energy \cite{buildingprot}. Geometrically, it remains identical for all climatic zones. However, construction parameters vary according to ASHRAE Standard 90.1 from 2019 \cite{ashrae9012019}, depending on climatic classifications. Other systems, such as ventilation, air-conditioning, and internal loads, are configured and sized following the former and other standards from ASHRAE. 

The application of PCM replaces the implemented thermal insulator, respecting its geometric dimensions. Thus, the thickness of the PCM is equal to that of the insulator, according to the original thermal parameters of the envelope. As a comparison criterion, simulations are performed where neither insulation nor PCM is applied to the surfaces, similar to a 2015 study \cite{melo2015thermal}. All simulations are executed using EnergyPlus, and the outputs are manipulated and analyzed using Python scripts, particularly with the Pandas library.

\subsection{Selection of representative cities}
According to ABNT NBR 15220-3 \cite{abnt15220}, Brazil is divided into eight bioclimatic zones (BZ). ASHRAE Standard 90.1 \cite{ashrae9012019} divides Brazil into four climatic regions: 0A (extremely hot humid), 1A (very hot humid), 2A (hot humid), and 3A (warm humid). However, the division of the entities is not reciprocal, that is, national climate classifications are not equivalent: the city of Rio de Janeiro, for example, according to ABNT, is located in bioclimatic zone 8 (most extreme), while the American Association classifies it in climatic region 1A. Most cities in climatic region 0A are also in bioclimatic zone 8. As a result, it was decided to prioritize the classification according to ABNT since it presents a more significant variability of climates and their characterization for defining construction guidelines. 

To evaluate the impact of the envelope in different climatic contexts, cities representing extreme bioclimatic zones 1 and 8, with ``antagonistic'' climates, are selected based on their population \cite{ibge}. Table \ref{tb:representative cities} condenses this information.

\begin{table}[H]
    \centering
    \begin{tabular}{@{} ccl @{}}
        \toprule
        ABNT Bioclimatic Zone & 
        ASHRAE Climatic Region & 
        Representative city \\
        \midrule
        1 & 3A & Curitiba, PR \\
        8 & 1A & Rio de Janeiro, RJ\\
        \bottomrule
    \end{tabular}
    \caption{Representative cities.}
    \label{tb:representative cities}
\end{table}%

It is worth stressing that, as a limitation of the present research, a direct extrapolation of the results for each of the cities to the other locations in their respective bioclimatic zones is not entirely feasible since, in zone 8, for example, the vast majority of cities are located in the north and northeast regions of Brazil thus they present a different solar chart than the city of Rio de Janeiro.

\subsection{The \textit{morphing} procedure and climate files}
The \textit{morphing} process of climate files is executed using the tool CCWorldWeatherGen, version 1.9 \cite{ccworldweathergen}. The free tool projects future climate scenarios based on the period from 1961 to 1990 \cite{belcher2005constructing} using data from the global circulation model HadCM3 (Hadley Center Coupled Model version 3), which models atmospheric and oceanic behavior on a global scale. It has a spatial resolution of approximately \(300\times \SI{300}{\kilo\metre}\), for which, given this low resolution, CCWorldWeatherGen performs a statistical process of dynamic downscaling and obtains the values of the climate variables for each location by averaging the results of four points on the HadCM3 model's spatial grid closest to the accurate coordinates of the desired location. 

The period considered in the model follows the World Meteorological Organization \cite{wmo} recommendations, which establish a 30-year interval as the definition of climate behavior. The organization also maintains the recommendation that the period between 1961 and 1990 is taken as a standard common reference in studies, projections, and monitoring of the climate and its anomalies. HadCM3 model data is made available by the IPCC \cite{ipcc} and the CCWorldWeatherGen tool employs the Panel's A2 emissions scenario to modify climate files. The cited scenario projects global economic development with regional growth focuses in a heterogeneous world with a growing population, in which fewer measures favoring sustainable development are effectively applied (yet, it is not the worst-case emissions outlook assessed by the IPCC). CCWorldWeatherGen transforms a .epw file (EnergyPlus Weather) based on the desired extrapolation for the representative years of 2020, 2050, and 2080, according to the projections obtained by the HadCM3 model, each characterizing a defined climate period of 30 years: 2010 to 2039, 2040 to 2069 and 2070 to 2099, respectively. 

Regarding climate data for building simulations in the selected cities, a paper from 2015 \cite{lamberts2015analise} studies three types of existing climate files for Brazilian cities: TRY (Test Reference Year), SWERA (Solar and Wind Energy Resource Assessment), and INMET (Brazil's National Institute of Meteorology). The cited work evaluated several meteorological variables from the archives. Through this, it was possible to identify that the INMET files had data acquisition failures in some periods, linked to possible problems in the automatic sensors. Regarding SWERA data, few Brazilian cities have climate files of this type. Therefore, TRY files become necessary. Files of this type consider a real year within a 10-year period, treating it as a reference year without temperature extremes. However, TRY files do not present complete information about solar radiation data, which is supplied through the Typical Meteorological Year or TMY \cite{crawley2015rethinking} file. In addition to having more information than TRY, it uses meteorological data from a longer interval, building a fictitious typical year, with months from different years within an analyzed period of 15 years. The authors discourage using TRY files and recommend using TMY. 

Given considerations and recommendations \cite{crawley2015rethinking, wmo}, one should opt for TMY-type files from older periods or a larger sample of years to ensure the reliability of the morphing process and the construction of realistic climate scenarios. In this sense, the authors of the present research chose to use the TMYx files, which construct typical meteorological years based on the entire existing collection period, made available by the Brazilian Building Labeling Program \cite{pbeedifica}. The original climate files' names are shown in Table \ref{tb:baseline climates}, and the data sampling period. 
\begin{table}[H]
    \centering
    \begin{tabular}{ @{} lll @{} }
        \toprule
            City & Climate file & Period \\
        \midrule
            \makecell[l]{Curitiba, PR \\ {}}& 
            \makecell[l]{\texttt{BRA\_PR\_Curitiba} \\ \texttt{-Pena.Intl.AP.838400\_TMYx.epw}} & 
            \makecell[l]{\text{1961-2017} \\ {}} \\[2.5ex]
            \makecell[l]{Rio de Janeiro, RJ \\ {}} & 
            \makecell[l]{\texttt{BRA\_RJ\_Rio.de.Janeiro} \\ \texttt{-Santos.Dumont.AP.837550\_TMYx.epw}} & 
            \makecell[l]{\text{1973-2017} \\ {}}\\
        \bottomrule
    \end{tabular}
    \caption{Base climate files for each representative city.}
    \label{tb:baseline climates}
\end{table}

However, it is important to highlight that the original climate files represent a typical meteorological year within the period described in Table \ref{tb:baseline climates}, considering all years in the sample. From the beginning of measurements until today, climate change has already affected the weather's behavior and given that the TMY encompasses a range of years with more prominent climate change effects -- the most recent ones seen in Table \ref{tb:tmyx months} -- the resultant typical year tends to incorporate the climatological profiles with this effects. Hence, it is possible to expect a slight overestimation of the effects of climate change when applying the \textit{morphing} methodology based on more recent climate files \cite{jentsch2013transforming, belcher2005constructing}, since the technique maintains the behavior of the original data. 

\begin{table}[H]
    \centering
    \begin{tabular}{ @{} lllllll @{} }
    \toprule
        City & Jan & Feb & Mar & Apr & May & Jun \\
        \midrule
        Curitiba & 1977 & 1987 & 1977 & 1964 & 1993 & 2008 \\
        Rio de Janeiro & 1989 & 1991 & 2017 & 1988 & 1981 & 1980 \\
        \midrule
        City & Jul & Aug & Sep & Oct & Nov & Dec \\
        \midrule
        Curitiba & 1980 & 1965 & 1986 & 2017 & 1977 & 1990 \\
        Rio de Janeiro & 1981 & 2001 & 1987 & 1987 & 2001 & 1979 \\
    \bottomrule
    \end{tabular}
    \caption{Reference years of the climate files used to construct the typical meteorological years.}
    \label{tb:tmyx months}
    \end{table}

\subsection{Energy model}
ASHRAE Standard 90.1 entitled ``Energy Standards for Buildings Except Low-Rise Residential Buildings'' \cite{ashrae9012019}, determines the guidelines for constructing and simulating thermoenergetic reference models, or baseline models, applicable for all climatic regions. The standard defines the thermal parameters of surfaces, minimum efficiency of heating and cooling systems, lighting densities, and HVAC systems suitable for construction typologies, according to the floor area and their locations. The energy model of this study is developed and made available, in its entirety, by the United States Department of Energy \cite{buildingprot}, through the Building Energy Codes Program (BECP), which, together with other typologies, is widely used in several studies, being structured in compliance with the aforementioned Standard 90.1. The set of prototypes represents about 75\% of the floor area of new buildings in all North American climates. 

This paper adopts the medium office building model (Figure \ref{fig:geometry}). This typology generally has integrated projects and systems with a greater focus on the application of thermal comfort and energy efficiency measures, besides, due to the building's operation characteristics, it is subject to greater thermal loads during critical periods. 

The models used in both bioclimatic zones are identical regarding the type of air-conditioning systems, internal loads, and building operation, differing in envelope parameters and HVAC system details, complying with ASHRAE Standard 90.1 \cite{ashrae9012019}. As a light steel-framed building, following the cited standard, it requires thermally insulated walls and roof, where Table \ref{tb:transmittance} lists the maximum thermal transmittance value (U-factor or just $U$) for this surfaces, according to the climatic regions. None of the mentioned parameters, with the exception of climatic conditions, are changed between simulations of the same bioclimatic zone. 
\begin{figure}[H]
    \centering
    \includegraphics[width=0.70\textwidth]{geometry}
    \caption{Building's geometric model.}
    \label{fig:geometry}
\end{figure}%
\begin{table}[H]
    \centering
    \begin{tabular}{@{} ccl @{}}
        \toprule
        \makecell[l]{Bioclimatic Zone\\(ASHRAE Climatic Region)} 
        & $U_{\text{roof}} \left[ \si{\watt\per\square\metre\per\kelvin} \right]$ 
        & $U_{\text{walls}} \left[ \si{\watt\per\square\metre\per\kelvin} \right]$ \\
        \midrule
        1 (3A) & 0.220 & 0.435 \\
        8 (1A) & 0.273 & 0.705\\
        \bottomrule
    \end{tabular}
    \caption{Thermal transmittance of the building surfaces.}
    \label{tb:transmittance}
\end{table}

The office building has three floors with 5 thermal zones each. These are classified as perimeter (1 to 4) or core (5), with the same definition of internal loads per unit area. Through Figure \ref{fig:building stories}, it is possible to visualize the model geometry in greater detail. The percentage of the opening area in all orientations is 33\%. The building's HVAC system is a single-duct, multi-zone variable air volume (VAV) type, where a fan with variable speed blows the mixed air (outdoor and return) into the building thermal zones, and each thermal zone has another fan capable of modulating the conditioned airflow via dampers, respecting the zone's temperature setpoints. The cooling and heating system is story-centralized, with reheating units in the thermal zones. 

In detail, each of the floors is served by a self-contained rooftop unit (or PACU, Packaged Air-Conditoning Unit) where the machine has, contained in a single housing, the entire air supply and conditioning system (heating and cooling coils) to serve the floors. Cooling is achieved through a direct expansion system, while the heating system is through the natural gas furnace, systems with efficiencies of, respectively, 340\% (COP) and 81\%. Each thermal zone has an individual heating system with an electric resistance at the terminal airflow units, with unitary efficiency. 

\begin{figure}[H]
    \centering
    \subfloat[\centering Orientation.]{{\includegraphics[width=5cm]{thermal zones} }}%}
    \qquad
    \subfloat[\centering Thermal zones and plenums.]{{\includegraphics[width=7cm]{building division} }}%}
    \caption{Detailing of the building stories.}
    \label{fig:building stories}
\end{figure}

Between the floors, there are return plenums for the air-conditioning systems that serve each of the respective floors and where, in practice, the ducts and fan boxes for each thermal zone are housed. Return air from all areas of the floor is blown into the space. In climate region 3A (ZB1, Curitiba), the air supply system on each floor has an economizer cycle, where the activation depends on the enthalpy differential between the outdoor air and the return air. If the enthalpy of the outdoor air is greater than that of the return air, the outdoor air flow rate is fixed at a minimum value. Otherwise, the dampers modulate the airflow rates to reach the temperature setpoint. This way, less energy is required to condition the supplied air. 

\subsection{Phase change material selection and simulation setup}
The energy model used has heating and cooling setpoint values of \SI{21}{\celsius} and \SI{24}{\celsius}, respectively, representing temperatures commonly used for activating HVAC systems. The selection of the phase change material respects these limits, aiming to store thermal energy at an appropriate temperature for both the HVAC system and the building's occupants, and thus, standardizing the material in different climates is possible. For this logic and taking as a reference the selected materials and results in some of the previously described works \cite{kalbasi2022buildings,carlucci2021phase,olivo2021impact}, the SP24E inorganic PCM, produced by RubiTherm \cite{sp24e}, was chosen, mainly because of the coincidentally material's operating range of temperatures and the HVAC setpoints. The manufacturer mentions that, for application in buildings, inorganic materials are generally preferred \cite{rubithermbuild}. The PCM in question presents other favorable characteristics for its selection: stable performance during phase change cycles, high latent heat per unit volume, limited subcooling, and low flammability, it is non-toxic and can be macroencapsulated. The PCM data used for the simulations can be seen in Table \ref{tb:sp24e data}. 

\begin{table}[H]
    \centering
    \begin{tabular}{@{} lll @{}}
        \toprule
        Property & Value & Dimension \\
        \midrule
        Melting temperature interval & 24-24 (peak at 24) & \si{\celsius} \\
        Freezing temperature interval & 23-21 (peak at 21) & \si{\celsius} \\
        Latent heat ($L$) & 150 & \si{\kilo\joule\per\kilo\gram} \\
        Specific heat ($c_p$) & 2 & \si{\kilo\joule\per\kilo\gram\kelvin} \\
        Solid state density ($\rho_{sol}$) & 1600 & \si{\kilo\gram\per\cubic\metre} \\
        Liquid state density ($\rho_{liq}$) & 1500 & \si{\kilo\gram\per\cubic\metre} \\
        Thermal conductivity (both phases) ($k$) & 0.5 & $\si{\watt\per\metre\kelvin}$ \\
        \bottomrule
    \end{tabular}
    \caption{SP24E data.}%
    \label{tb:sp24e data}
\end{table}%

The selected PCM is modeled in EnergyPlus using class \textit{MaterialProperty:PhaseChangeHisteresys}, capable of modeling the cyclic behavior of the latent phase of materials. The class requests properties of both physical states, described in Table \ref{tb:sp24e data}. To do so, it is necessary to use the CondFD finite difference algorithm. Due to its robustness and stability \cite{al2019application}, the fully implicit scheme is used, such as the TARP algorithms for internal convective exchanges and the DOE-2 for external convective exchanges. As discretization parameters, the values of 0.3 for the space discretization constant and 60 time steps per hour, the maximum allowed by EnergyPlus, were used \cite{tabares2012verification}. Such algorithms and configurations are not necessary for simulating envelopes without PCM but are maintained in favor of standardizing the numerical methodology between simulations. Tolerances for numerical convergences were kept as the software default values. 

The thickness of the PCM used on each surface is equivalent to the thickness of the insulation applied in each of the respective façade systems since the aim is to replace one with the other geometrically. According to the models \cite{buildingprot}, the thermal resistances ($R$) of the insulators used in each envelope are visible in Table \ref{tb:thickness}.

It should be noted that, in the energy models, insulators are represented only through their thermal resistance values, without the definition of physical material, that is, without thermal mass. Fiberglass, mineral wool, and cellulose are among the most common thermal insulation materials used in civil construction \cite{insulation}, which have a similar thermal conductivity ($k$) and can be approximated by \SI{0.04}{{\watt\per\metre\kelvin}} \cite{jones2017performance}. The theoretical thickness ($e$) of the insulating material is given by multiplying the thermal resistance ($R$) and the thermal conductivity of the same material ($k$). This thickness is also the thickness of the PCM that will replace the insulator. 
\begin{table}[H]
    \centering
    \begin{tabular}{@{} ccccc @{}}
        \toprule
        & \multicolumn{2}{c}{Roof} & 
        \multicolumn{2}{c}{External Wall} \\
        \cmidrule(lr){2-3} \cmidrule(ll){4-5}
        Bioclimatic Zone & 
        $R \left[ \si{\square\metre\kelvin\per\watt} \right]$ & 
        $e \left[ \si{\centi\metre} \right]$ & 
        $R \left[ \si{\square\metre\kelvin\per\watt} \right]$ & 
        $e \left[ \si{\centi\metre} \right]$\\
        \midrule
        1 (3A) & 4.319 & 17 & 1.903 & 8 \\
        8 (1A) & 3.472 & 14 & 1.037 & 4 \\
        \bottomrule
    \end{tabular}%   
    \caption{Thermal resistance used in each model and material 
    thickness.}%
    \label{tb:thickness}%
\end{table}

Table \ref{tb:thickness} presents the thicknesses of the PCM applied to each surface according to its climate zone in integer values, given the need for discrete encapsulation. Figure \ref{fig:sections} illustrates the generic constructions of the roof and wall surfaces, highlighting the layer of material -- insulation or PCM -- that is changed between simulations.
\begin{figure}[H]
    \centering
    \subfloat[\centering Roof section.]{{\includegraphics[width=6cm]{roof section} }}%}
    \qquad
    \subfloat[\centering Wall section.]{{\includegraphics[width=6cm]{wall section} }}%}
    \caption{Detailing of surface constructions, out of scale.}
    \label{fig:sections}%
\end{figure}

In the case of the non-insulated simulated envelopes, it is important to stress that it is not a cavity with a void between its layers: for this research, an uninsulated envelope is a construction with the absence of insulating material, and the adjacent layers are close to each other, without any gaps.

\section{Results and discussion}
A total of 18 simulations were carried out: 9 simulations for each bioclimatic zone, divided into simulations varying the composition of the envelope (resistive thermal insulation, phase change material, and uninsulated) for each of the climate files (original TMYx, 2050 projection, and 2080 projection). The \textit{morphing} process was executed using the CCWorldWeatherGen tool, generating climate files for the representative years 2050 and 2080 from the TMYx. The average duration of simulations using EnergyPlus with insulated or uninsulated surfaces was approximately 30 to 35 minutes, while the phase change material simulations lasted for an average time of roughly 40 to 45 minutes. The simulations were performed using an i5-8250U processor and 8 GB of DDR4 RAM. 

Metrics for the entire building and representative thermal zones are evaluated. The thermal zones and spaces selected for analysis are the zone 1 (south façade) and zone 3 (north façade), on the middle floor, both occupied and air-conditioned and the return plenum on the roof. Inspired by the literature \cite{kishore2021enhancing}, the monthly net heat balance $Q_{net}^m$ is presented for each of the simulated climate behaviors, to evaluate, particularly, the impact of façades' orientation: it expresses the sum of the hourly-mean conduction heat flux on the internal face of the surface $q^{\prime\prime}_{int}[h]$ for a given month $m$, during occupation hours

\begin{equation}
    Q_{net}^{m}=\sum_{H hours}\,\frac{q^{\prime\prime}_{int}[h]}{1000}
    \label{eq 16}
\end{equation}

The metric results in monthly heat gain or loss during the hours of building and HVAC system operation. The hourly data of mean radiant temperature, operative temperature, and heat flux through the envelope are assessed during typical weeks defined by the TMYx climate files (weeks with mean temperatures closest to the mean for each season). Unlike the operative temperature, we chose to use the mean radiant temperature to evaluate thermal zones 1 and 3, as the environments are conditioned. Therefore, the average air temperature approaches the setpoint and does not illustrate the direct impact of the envelope as desired. In parallel, the operating temperature of the plenum is investigated, as this environment only has opaque surfaces and is not conditioned but receives return air from the thermal zones.

Emphasizing the phase change materials applications, the PCM's physical state is illustrated on an hourly basis through two behaviors defined as ``characteristic'' and ``predominant''. EnergyPlus generates an output for each node's state comprising a continuous value between -2 (liquid) and 2 (solid). Utilizing the Mean Value Theorem, a representative (characteristic) numerical value for the whole material is extracted from Simpson's Integration Rule using all node values. The predominant state is the mode of the nodal results. 

Building-wise, the total monthly energy consumption with the modeled HVAC system is extracted and divided by the conditioned floor area, a metric represented by Energy Use Intensity, or EUI, intending to incorporate the partial load profile that air-conditioning equipment can have. Tables \ref{tb:annual EUI curitiba} and \ref{tb:annual EUI rio} summarize the results of the annual EUI with the HVAC systems for each of the envelopes in the simulated climate scenarios. The acronyms ``NOM'', ``INS'', and ``PCM'' represent the three envelope configurations: uninsulated (no material applied), resistive thermal insulator, and phase change material, respectively.

\begin{table}[H]
\centering
    \begin{tabular}{@{} lccc @{}}
        \toprule
        Simulated climate & 
        NOM $\left[ \si{\kWh\per\square\metre} \right]$ & 
        INS $\left[ \si{\kWh\per\square\metre} \right]$ & 
        PCM $\left[ \si{\kWh\per\square\metre} \right]$ \\
        \midrule
        TMYx & 37.17 & 21.17 & 20.35 \\
        2050 Projection & 40.49 & 29.15 & 26.76 \\
        2080 Projection & 48.99 & 35.93 & 32.34 \\
        \bottomrule
    \end{tabular}
    \caption{Annual Energy Use Intensity (EUI) in Curitiba, PR (BZ1/3A).}
    \label{tb:annual EUI curitiba}
\end{table}%

\begin{table}[H]
\centering
    \begin{tabular}{@{} lccc @{}}
        \toprule
        Simulated climate & 
        NOM $\left[ \si{\kWh\per\square\metre} \right]$ & 
        INS $\left[ \si{\kWh\per\square\metre} \right]$ & 
        PCM $\left[ \si{\kWh\per\square\metre} \right]$ \\
        \midrule
        TMYx & 52.38 & 41.27 & 38.15 \\
        2050 Projection & 69.23 & 50.86 & 57.03 \\
        2080 Projection & 80.57 & 58.38 & 70.48 \\
        \bottomrule
    \end{tabular}
    \caption{Annual Energy Use Intensity (EUI) in Rio de Janeiro, RJ (BZ8/1A).}
    \label{tb:annual EUI rio}
\end{table}%

The results demonstrate lower EUI values for the PCM envelope compared to the others in bioclimatic zone 1 in all simulated climates. For bioclimatic zone 8, the envelope with phase change material shows the lowest EUI value for the typical meteorological year. Still, it exhibits larger results compared to the envelope with a resistive thermal insulator in the climate projections simulations. The envelope without insulating material, with the highest thermal transmittance, presents the highest EUI values in all simulations. 

\begin{table}[H]
    \centering
        \begin{tabularx}{\textwidth}{@{} X 
            *3{S[table-format=+1.3,round-precision=3]} 
            *3{S[table-format=+2.3,round-precision=3]}
            @{}}
            \toprule
             & \multicolumn{3}{c}{Curitiba, PR (BZ1/3A)} 
             & \multicolumn{3}{c@{}}{Rio de Janeiro, RJ (BZ8/1A)} 
             \\
            \cmidrule(lr){2-4}		\cmidrule(ll){5-7}
            Climate & 
                \multicolumn{1}{r}{NOM} &
                \multicolumn{1}{r}{INS} &
                \multicolumn{1}{r}{PCM} &
                \multicolumn{1}{r}{NOM} &
                \multicolumn{1}{r}{INS} &
                \multicolumn{1}{r@{}}{PCM}
                \\
            \midrule
            TMYx & -0.112 & 0.597 & -0.763 & -0.410 & -0.282 & 0.794 \\
            2050 projection & -0.134 & 0.271 & -0.121 & -0.282 & -0.125 & 0.815 \\
            2080 projection & -0.282 & 0.085 & -0.353 & -0.287 & -0.135 & 0.443 \\
            \bottomrule
        \end{tabularx}%
        \caption{Pearson Correlation Coefficients between monthly HVAC EUI and envelope net heat balance: north façade}
        \label{tb:pearson north}
    \end{table}

\begin{table}[H]
\centering
\begin{tabularx}{\textwidth}{@{} X 
    *3{S[table-format=+1.3,round-precision=3]} 
    *3{S[table-format=+2.3,round-precision=3]}
    @{}}
    \toprule
        & \multicolumn{3}{c}{Curitiba, PR (BZ1/3A)} 
        & \multicolumn{3}{c@{}}{Rio de Janeiro, RJ (BZ8/1A)} 
        \\
    \cmidrule(lr){2-4}		\cmidrule(ll){5-7}
    Climate & 
        \multicolumn{1}{r}{NOM} &
        \multicolumn{1}{r}{INS} &
        \multicolumn{1}{r}{PCM} &
        \multicolumn{1}{r}{NOM} &
        \multicolumn{1}{r}{INS} &
        \multicolumn{1}{r@{}}{PCM}
        \\
    \midrule
    TMYx & -0.144 & 0.963 & -0.227 & 0.922 & 0.885 & 0.710 \\
    2050 projection & 0.945 & 0.960 & -0.957 & 0.949 & 0.918 & 0.940 \\
    2080 projection & 0.967 & 0.956 & -0.954 & 0.944 & 0.906 & 0.973 \\
        \bottomrule
\end{tabularx}%
\caption{Pearson Correlation Coefficients between monthly HVAC EUI and envelope net heat balance: south façade}
\label{tb:pearson south}
\end{table}

\begin{table}[H]
\centering
    \begin{tabularx}{\textwidth}{@{} X 
        *3{S[table-format=+1.3,round-precision=3]} 
        *3{S[table-format=+2.3,round-precision=3]}
        @{}}
        \toprule
            & \multicolumn{3}{c}{Curitiba, PR (BZ1/3A)} 
            & \multicolumn{3}{c@{}}{Rio de Janeiro, RJ (BZ8/1A)} 
            \\
        \cmidrule(lr){2-4}		\cmidrule(ll){5-7}
        Climate & 
            \multicolumn{1}{r}{NOM} &
            \multicolumn{1}{r}{INS} &
            \multicolumn{1}{r}{PCM} &
            \multicolumn{1}{r}{NOM} &
            \multicolumn{1}{r}{INS} &
            \multicolumn{1}{r@{}}{PCM}
            \\
        \midrule
        TMYx & -0.150 & 0.930 & -0.941 & 0.920 & 0.856 & -0.969 \\
        2050 projection & 0.945 & 0.950 & -0.984 & 0.964 & 0.928 & 0.923 \\
        2080 projection & 0.971 & 0.950 & -0.988 & 0.963 & 0.921 & 0.983 \\
            \bottomrule
    \end{tabularx}%
    \caption{Pearson Correlation Coefficients between monthly HVAC EUI and envelope net heat balance: roof surface}
    \label{tb:pearson roof}
\end{table}

Tables \ref{tb:pearson north}, \ref{tb:pearson south}, and \ref{tb:pearson roof} show the Pearson Correlation Coefficients between the monthly Energy Use Intensity and the net heat balance across each of the envelope surfaces evaluated. 

\subsection{Results for Bioclimatic Zone 1/ASHRAE 3A (Curitiba, PR)}
Through Figure \ref{fig:eui curitiba}, it is possible to observe the monthly behavior of the Energy Use Intensity with the HVAC systems for the city of Curitiba. Considering the typical meteorological year, it is necessary to use heating, especially in the case of the envelope without any material.
\begin{figure}[H]
    \centering
    \includegraphics[width=1\textwidth]{EUI comp Curitiba}
    \caption{Monthly behavior of the EUI with HVAC systems in Curitiba, PR (ZB1/3A)}
    \label{fig:eui curitiba}
\end{figure}

Highlighting in Figure \ref{fig:eui curitiba}, consumption in winter months (June to September) exceeds consumption in the summer months (December to March). This can be justified by using the heating system through individual thermal resistances at the thermal zones' ventilation terminals. Such equipment has unitary efficiency, making it less efficient than the direct expansion used to cool environments. The central heating system is also required during the winter, increasing energy consumption. %revisar

In comparison with other envelopes, the application of PCM results in an annual EUI that is approximately 82.5\% lower during the typical year and, on average, 51.4\% in future climates. The envelope with thermal insulation, in turn, presents an annual EUI, during the TMYx simulation, 75.4\% lower than the envelope without material, and the average in future climates is 37.6\% lower. When comparing the insulated envelopes with each other, the application of phase change material produces lower annual EUI: around 3.9\% lower in the typical year, 8.2\% in 2050, and 10\% in 2080. 

Due to temperature increases in climate projections, there is a change in the behavior of energy consumption during the winter period for the building without insulating material in the envelope. In both simulated projections, the winter months have the lowest EUI value of the year, that is, the thermal demand for heating is ``converted'' into thermal demand for cooling. Regarding simulations in which the envelope has either thermal insulation or phase change material, the monthly energy consumption profile is similar and very close throughout the year for the TMYx climate. The ``detachment'' of the consumption curves is notable in the months outside the winter period in future climate projections, with a difference of almost \SI{1}{\kWh\per\square\metre} in 2080. 

\subsubsection{North façade (Thermal Zone 3)}
Figure \ref{fig:net heat north curitiba} illustrates the net heat balance on the inner face of the surface ($Q_{net}$), resulting in monthly heat gains by the thermal zone when the balance is positive or heat losses by the zone when the balance is negative. It is observed that, for all simulations, the heat balance on the surface with the phase change material is always negative, denoting a net heat loss from the thermal zone during its occupation, which reduces the monthly cooling load. During the winter months, especially in Figure \ref{fig:eui curitiba} (a), energy consumption for this envelope decreases, which does not mean that there is a demand for heating, but instead that the thermal zone, through solar gains and internal loads, together with thermal insulation, manages to maintain comfortable temperature levels passively. In this sense, the temperature of the internal face can remain close to the material's solidification temperature. In contrast, its core and, mainly, the external face of the façade have lower temperatures, which reduces the heat transfer rate through this surface.

\begin{figure}[H]
    \centering
    \includegraphics[width=1\textwidth]{zn3 curitiba}
    \caption{Monthly net heat balance across the north façade in Curitiba, PR (ZB1/3A)}
    \label{fig:net heat north curitiba}
\end{figure}

The envelope without any material does not show characteristic behavior during the typical meteorological year, having one of the lowest values for the heat balance in June (\SI{-2.15}{\kWh\per\square\metre}), the month with the highest EUI of \SI{3.93}{\kWh\per\square\metre}. In October, outside the intense cold season, during spring, this envelope has the lowest value for the heat balance (\SI{-2.86}{\kWh\per\square\metre}), and, consequently, one of the highest EUI recorded for this envelope (\SI{2.95}{\kWh\per\square\metre}). The absence of insulating material causes intense heat gains in the future climates' simulations, frequently exceeding \SI{5}{\kWh\per\square\metre} and \SI{7.5}{\kWh\per\square\metre} per month, reaching maximums of \SI{8.10}{\kWh\per\square\metre} in March 2050 and \SI{10.42}{\kWh\per\square\metre} in May 2080, possibly due to the intensification of direct solar irradiation during this period. The application of the resistive thermal insulator guarantees the maintenance of low levels of heat gains and losses throughout the year, as observed in all simulations. 

\begin{figure}[H]
    \centering
    \includegraphics[width=1\textwidth]{tq zn3 curitiba}
    \caption{Hourly curves of Mean Radiant Temperature in Thermal Zone 3 and Heat Flow by conduction on the internal face of the north façade during the typical winter week in Curitiba, PR (ZB1/3A).}
    \label{fig: temps and hflux north curitiba}
\end{figure}

The hourly curves of mean radiant temperature ($T_{mr}$) and conduction heat flux on the inner face of the envelope surface ($q^{\prime\prime}_{int}$) for Thermal Zone 3 are expressed in Figure \ref{fig: temps and hflux north curitiba}. The positive direction of heat flux indicates a flow from the outside to the inside of the thermal zone. For Curitiba, we chose to study the profile of the variables in the typical winter week, defined between July 8th and 15th, according to the location's TMYx climate file.

Simulations with phase change material in the envelope, for the typical meteorological year, result in mean radiant temperatures higher than other envelopes, especially among the working periods on weekdays, with a weekly average of \SI{24.06}{\celsius}. During the typical week (Figure \ref{fig: temps and hflux north curitiba} (a)), at night, conductive heat flow in the envelope with PCM is positive or very close to zero, contrasting with other applications, in which the heat flow is negative. In this sense, Thermal Zone 3 loses heat at a lower rate or even receives heat in this interval of hours through the external wall. This passive behavior is complemented by the heat removal during the day due to the PCM, reducing the temperature range. Thus, when the building returns to operation the following day, the thermal zone presents a higher mean radiant temperature. Consequently, its thermal heating load decreases, especially at the beginning of occupancy. The material stores heat from the thermal zone during the day and releases it back during the night. The weekly mean radiant temperature is \SI{24.85}{\celsius} in the 2050 projection and \SI{25.14}{\celsius} in 2080 over the occupied period. 

For the 2050 and 2080 climate projections -- in Figure \ref{fig: temps and hflux north curitiba} (b) and (c) -- the thermal zone temperatures with insulation and with PCM become practically identical, with few variations throughout the typical week, supporting the similar EUI behaviors visible in Figure \ref{fig:eui curitiba} (b), in the months of June and July. Under these climatic scenarios, the weekly mean radiant temperature is \SI{24.98}{\celsius} and \SI{25.46}{\celsius}, respectively, for the insulated envelope. It is noteworthy the façade's greater incidence of direct solar radiation during winter, which is translated by the higher heat flux when uninsulated. In an antagonistic way, the reduction in external temperatures during the night produces temperature gradients of almost \SI{6}{\celsius} for this envelope, which becomes much more sensitive to weather oscillations, presenting averages of \SI{22.09}{\celsius} in the typical year and \SI{24.32}{\celsius} and \SI{25.20}{\celsius} in the 2050 and 2080 climates, which corroborates the change in the EUI pattern presented by the building with this envelope. 

\begin{figure}[H]
    \centering
    \includegraphics[width=1\textwidth]{pcm zn3 curitiba}
    \caption{Hourly behavior of the PCM's physical state applied to the north façade during a typical winter week in Curitiba, PR (ZB1/3A).}
    \label{fig:pcm north curitiba}
\end{figure}

The behavior of the heat balance across the façade is complemented by the analysis of the characteristic physical state of the phase change material through Figure \ref{fig:pcm north curitiba}. When observing all the curves, the cyclical behavior of the physical state is notable, explaining the daily completeness of the thermal hysteresis process by the PCM on this surface: during the beginning of the day, the material is solid, releasing the thermal energy to the surroundings. Over time, the material spontaneously removes energy from its surroundings, melting. The PCM's thermal hysteresis cycle occurs within the air-conditioning system's setpoint range, therefore, exothermic and endothermic processes regulate the temperature of the internal face to levels consistent with the system. 

\subsubsection{South façade (Thermal Zone 1)}
When looking at Figure \ref{fig:net heat south curitiba}, as in Thermal Zone 3, for the typical meteorological year, Figure \ref{fig:net heat south curitiba} (a), and for the 2050 climate, (b), the thermal zone loses heat intensely, with a minimum of approximately \SI{-6}{\kWh\per\square\metre} in June of the TMY. Likewise, as in Thermal Zone 3, heat loss through the wall -- in cases with PCM -- decreases in the winter months, denoting the reduction in demand for spontaneous cooling of the thermal zone. 

\begin{figure}[H]
    \centering
    \includegraphics[width=1\textwidth]{zn1 curitiba}
    \caption{Monthly net heat balance across the south façade in Curitiba, PR (ZB1/3A)}
    \label{fig:net heat south curitiba}
\end{figure}

Due to heat loss through the envelope on the south façade when no insulating material is applied, the heating load overlaps the cooling load, justifying the high-EUI period for this application during colder months. Annual heat losses during the typical meteorological year, reach almost \SI{-49}{\kWh\per\square\metre}. During winter, the envelopes with phase change material and thermal insulation present similar EUI and net heat balance on the south-oriented wall. Outside this period, stressing the climate projections, the PCM envelope has a better capacity to remove heat from the thermal zone, reducing the cooling load: annually, the PCM façade presents average heat losses of \SI{-16.45}{\kWh\per\square\metre}, while the thermal insulator is \SI{-7.28}{\kWh\per\square\metre}, around 56\% lower. The impact of the absence of solar radiation is easily observed through the monthly heat losses through the façade without insulating material. 

\begin{figure}[H]
    \centering
    \includegraphics[width=1\textwidth]{tq zn1 curitiba}
    \caption{Hourly curves of Mean Radiant Temperature in Thermal Zone 1 and Heat Flow by conduction on the internal face of the south façade during the typical winter week in Curitiba, PR (ZB1/3A).}
    \label{fig: temps and hflux south cuitiba}
\end{figure}

In Figure \ref{fig: temps and hflux south cuitiba}, as occurs for the north façade during the typical meteorological year (a) and for the 2050 projection (b), the conductive heat flux for the PCM envelope remains negative. Thus the area loses heat to the wall with the material. The mean radiant temperature of Thermal Zone 1, in the TMY simulation with PCM, is higher all week when compared to the other cases, reaching a weekly average of \SI{23.23}{\celsius} during occupancy. Similar to Thermal Zone 3, the heat flux on the inner face of the envelope is positive or null during most of the night hours of the week, especially in Figure \ref{fig: temps and hflux south cuitiba} (a). Due to the storage of thermal energy by the PCM, such passive heating guarantees lower mean radiant temperature gradients. For this thermal zone, the PCM is capable of maintaining the mean radiant temperature during the week between \SI{23}{\celsius} and \SI{24.5}{\celsius} in all simulated climates. The envelope without insulating material doesn't show an average $T_{mr}$ above \SI{24}{\celsius} in any of the simulations during working hours, with the exception of 2080, which registers the highest among the proposed envelopes, equal to \SI{24.57}{\celsius}. 

Regarding the physical state of the PCM, shown in Figure \ref{fig:pcm south curitiba}, for the climate described by the TMYx file (Figure \ref{fig:pcm south curitiba} (a)), the predominance of the solid or partially solid state can be noted during much of the typical winter week, maintaining an intermediate state. This can be justified by the low heat fluxes during this interval, given the low outdoor temperature and the proximity of the internal temperature to the PCM operating point. In this sense, the core of the material functions as a barrier for heat transfer between the outdoor environment and the internal environment during phase change. 

When considering climate changes in the simulations, the PCM presents more evident daily hysteresis cycles, especially as exposed in Figure \ref{fig:pcm south curitiba} (c), indicating that the performance of the material improves in future climates for this façade, which also supports the heat removal intensification from the thermal zone in these simulations. 

\begin{figure}[H]
    \centering
    \includegraphics[width=1\textwidth]{pcm zn1 curitiba}
    \caption{Hourly behavior of the PCM's physical state applied to the south façade during a typical winter week in Curitiba, PR (ZB1/3A).}
    \label{fig:pcm south curitiba}
\end{figure}

\subsubsection{Roof (Plenum)}
Due to the thickness of the material -- insulation or PCM -- in the roof surface being \SI{17}{\centi\metre}, the heat flow is extremely low, resulting in monthly heat balances close to zero in these applications, as can be seen in Figure \ref{fig:net heat roof curitiba}. Compared to the non-insulated envelopes, the shear scale difference of these heat balances is also noteworthy. 

\begin{figure}[H]
    \centering
    \includegraphics[width=1\textwidth]{plenum curitiba}
    \caption{Monthly net heat balance across the roof façade in Curitiba, PR (ZB1/3A)}
    \label{fig:net heat roof curitiba}
\end{figure}

In contrast, the absence of insulating material makes heat transfer across the surface much more sensitive to temperature gradients. Comparing the envelope in question with those that have a material applied, it is visible that the plenum exchanges heat with the outdoor environment intensely, particularly, at the height of summer and winter: for the typical year, as the envelopes with PCM and insulation present, respectively, annual heat gains of \SI{2.12}{\kWh\per\square\metre} and \SI{4.57}{\kWh\per\square\metre} and annual heat losses of \SI{-3.07}{\kWh\per\square\metre} and \SI{-5.11}{\kWh\per\square\metre}, whereas the roof without any material applied records annual gains of \SI{47.53}{\kWh\per\square\metre} and losses of \SI{-34.82}{\kWh\per\square\metre}. 

\begin{figure}[H]
    \centering
    \includegraphics[width=1\textwidth]{tq plenum curitiba}
    \caption{Hourly curves of Operative Temperature in the plenum and Heat Flow by conduction on the internal face of the roof façade during the typical winter week in Curitiba, PR (ZB1/3A).}
    \label{fig:temps and hflux roof curitiba}
\end{figure}

Between May and August of the typical year, the building with an uninsulated roof records the highest heat losses, with a minimum of \SI{-3.68}{\kWh\per\square\metre} in June and the highest EUI values with the HVAC system in the same month. The low plenum temperature directly affects the supply air –- a mixture of the return air in the plenum and outdoor air –- and demands more thermal energy from the heating coils to reach the thermal zone's setpoint. Still, in the non-insulated roof case, in certain months of autumn (April and May) and spring (September and October), during the TMY, and also June and July of the climate 2050 climate, net heat balance through the surface is close to zero, justified by the thermal balance between the surface faces (internal and external), solar radiation intensity and return air temperature in the plenum. The significant heat gains in the summer period stand out for both climate projections, exceeding \SI{5}{\kWh\per\square\metre}. 
\begin{figure}[H]
    \centering
    \includegraphics[width=1\textwidth]{pcm plenum curitiba}
    \caption{Hourly behavior of the PCM's physical state applied to the roof surface during a typical winter week in Curitiba, PR (ZB1/3A).}
    \label{fig:pcm roof curitiba}
\end{figure}

From Figure \ref{fig:pcm roof curitiba}, the PCM envelope can maintain the operating temperature of the space almost constant throughout the evaluated week for all simulated climates, keeping its average below \SI{24}{\celsius}, during working hours. Even during the typical winter week, the uninsulated envelope causes high operating temperatures and large thermal amplitudes, with variations of almost \SI{17}{\celsius} daily. The heat fluxes through the roof, in this application, reinforce this gradient-intense behavior.

The characteristic physical state of the PCM on the roof -- Figure \ref{fig:pcm roof curitiba} -- shows phase change cycles with greater definition compared to the material's behavior on the south façade, but less clear than on the north façade. The thickness of the material, \SI{17}{\centi\metre}, may prevent its entire volume from changing phase in a uniform manner, whose hysteresis cycles are incomplete, and a form of physical state stratification is developed. The behavior of the characteristic physical state varies within the phase transition region and with a tendency to solidify, especially for the simulation with the TMYx file. In this climatic situation, the volume of material mostly does not reach the liquid state.

\subsection{Results for Bioclimatic Zone 8/ASHRAE 1A (Rio de Janeiro, RJ)}

Naturally, when considering the bioclimatic zone 8 climate, the studied typology exhibits higher monthly EUI values compared to the climate examined earlier (refer to Figure \ref{fig:eui rio de janeiro}).

\begin{figure}[H]
    \centering
    \includegraphics[width=1\textwidth]{EUI comp rj}
    \caption{Monthly behavior of the EUI with HVAC systems in Rio de Janeiro, RJ (ZB8/1A)}
    \label{fig:eui rio de janeiro}
\end{figure}

However, in the climate forecasts, the building's EUI is quite similar when using phase change material compared to the uninsulated envelope. Annually, it's around 12.1\% higher in 2050 and 20.7\% higher in 2080 than the EUI of the envelope with resistive thermal insulation, exposing a notable performance discrepancy between applications. Notably, the most significant differences between the two envelopes occur during specific months: January to May, November, and December. During the winter and spring period, in the climate of 2050 (as shown in Figure \ref{fig:eui rio de janeiro} (b)), using PCM results in the lowest EUI values. This effect, however, is only observed during the winter for the 2080 projection, as illustrated in Figure \ref{fig:eui rio de janeiro} (c). Furthermore, when comparing the annual EUI, the envelope with resistive thermal insulation presents reductions of 21.2\%, 26.5\%, and 27.5\% for the simulated climates, respectively, compared to the uninsulated building.

\subsubsection{North façade (Thermal Zone 3)}

Observing Figure \ref{fig:net heat north rio}, the absence of insulating material, again, causes high monthly heat gains throughout the year, resulting in \SI{95.01}{\kWh\per\square\metre}, \SI{131.11}{\kWh\per\square\metre}, and \SI{143.19}{\kWh\per\square\metre} annually for the simulated climates. 

\begin{figure}[H]
    \centering
    \includegraphics[width=1\textwidth]{zn3 rj}
    \caption{Monthly net heat balance across the north façade in Rio de Janeiro, RJ (ZB8/1A)}
    \label{fig:net heat north rio}
\end{figure}

The phase change material produces monthly heat losses between May and November of the TMY, shown in Figure \ref{fig:net heat north rio} (a), the period in which the most significant difference in monthly EUI occurs for this application compared to the others. The insulating material can maintain a uniform and similar profile -- with differences in magnitude -- among all climates. For climate projections, heat gains through the north façade with PCM exceed those of the same façade with resistive thermal insulation: for the 2050 projection, they are 57.86\% greater, whereas for the 2080 climate projection, they reach 113.75\% more. 

The typical week defined for Rio de Janeiro, according to the TMYx file, is from December 24 to 31, presenting an average temperature close to the annual average. Since the city's climate does not show significant differences in temperature throughout the seasons, the file does not specify a typical seasonal week. Visibly through Figure \ref{fig:temps and hflux north rio} (b) and (c), the mean radiant temperatures and heat fluxes through the façade with PCM are higher than the thermal insulation ones -- the mean radiant temperatures are around \SI{0.3}{\celsius} higher -- supporting the higher EUI consumption in this case during December for climate projections, a behavior opposite to that presented for the same envelope in the typical meteorological year. In this simulated climate, the mean radiant temperature for the case with insulation is about \SI{0.4}{\celsius} higher than for the case with PCM. 

\begin{figure}[H]
    \centering
    \includegraphics[width=1\textwidth]{tq zn3 rj}
    \caption{Hourly curves of Mean Radiant Temperature in Thermal Zone 3 and Heat Flow by conduction on the internal face of the north façade during the typical week in Rio de Janeiro, RJ (ZB8/1A).}
    \label{fig:temps and hflux north rio}
\end{figure}

The characteristic physical state of the phase change material, evaluated for the typical week, in the climate projections of 2050 (Figure \ref{fig:pcm north rio} (b)) and 2080 (Figure \ref{fig:pcm north rio} (c)) show a constantly liquid physical state, justifying December's EUI and other summer months, as well as the heat flow and temperature behaviors exposed in Figure \ref{fig:temps and hflux north rio}. 

\begin{figure}[H]
    \centering
    \includegraphics[width=1\textwidth]{pcm zn3 rj}
    \caption{Hourly behavior of the PCM's physical state applied to the north façade during a typical week in Rio de Janeiro, RJ (ZB8/1A).}
    \label{fig:pcm north rio}
\end{figure}

Given the high temperatures, even during the typical meteorological year, the high frequency of physical state changes during the day can be observed, especially between the 27th and 28th. Possibly, proximity between internal and external temperatures occurs, which also reduces the heat flow by conduction across the surface's core. This temperature trend maintains a significant volume of the PCM in a transitional physical state, subject to state variations more easily. 

\subsubsection{South façade (Thermal Zone 1)}

The behavior of the monthly net heat balance for the south façade resembles the monthly energy consumption illustrated in Figure \ref{fig:net heat south rio}, especially when observing the envelope with phase change material: as the monthly net heat balance denotes loss of thermal energy, the monthly EUI value for the respective building becomes the lowest among the three simulated envelopes, with emphasis on the future climates. As seen in Table \ref{tb:pearson south}, the Pearson Correlation Coefficients for the 2050 and 2080 climate projections are 0.940 and 0.973, respectively, denoting a strong correlation between the variables. 

\begin{figure}[H]
    \centering
    \includegraphics[width=1\textwidth]{zn1 rj}
    \caption{Monthly net heat balance across the south façade in Rio de Janeiro, RJ (ZB8/1A)}
    \label{fig:net heat south rio}
\end{figure}

Similar to Thermal Zone 3, Thermal Zone 1, observing Figure \ref{fig:temps and hflux south rio}, shows higher mean radiant temperature results on the façade with phase change material than those presented by the wall with thermal insulation in climate projections. The weekly average for this application is \SI{26.09}{\celsius} and \SI{26.26}{\celsius} for 2050 and 2080. In the case of the insulted envelope, the average in these periods reduces to \SI{25.78}{\celsius} and \SI{25.90}{\celsius}, all during occupied periods. 

\begin{figure}[H]
    \centering
    \includegraphics[width=1\textwidth]{tq zn1 rj}
    \caption{Hourly curves of Mean Radiant Temperature in Thermal Zone 1 and Heat Flow by conduction on the internal face of the south façade during the typical week in Rio de Janeiro, RJ (ZB8/1A).}
    \label{fig:temps and hflux south rio}
\end{figure}

As per the data presented in Figures \ref{fig:eui rio de janeiro} and \ref{fig:net heat south rio} (b) and (c), it is perceivable that in the future climates, the energy consumption and net heat balance in December show higher values for the envelope incorporating phase change material in comparison to the simulations with insulated envelope, particularly in December. Factors such as the elevated external temperatures and the thickness of the \SI{4}{\centi\metre} material, in contrast to other applications, contribute to the phase change material maintaining the liquid state throughout Rio de Janeiro's typical week, as depicted in Figure \ref{fig:pcm south rio} (b) and (c). This persistent state increases the thermal transmittance of the surface, which explains the higher energy consumption and net heat gains in this scenario.

\begin{figure}[H]
    \centering
    \includegraphics[width=1\textwidth]{pcm zn1 rj}
    \caption{Hourly behavior of the PCM's physical state applied to the south façade during a typical week in Rio de Janeiro, RJ (ZB8/1A).}
    \label{fig:pcm south rio}
\end{figure}

\subsubsection{Roof (Plenum)}

Similar to other surfaces, the monthly net heat balance across the roof corroborates the behavior of monthly energy consumption with air-conditioning -- see Figure \ref{fig:net heat roof rio}. The non-insulated roof allows significant heat gains to the return plenum during the year, with values considerably higher than those presented by other envelope compositions. For this envelope, Pearson Correlation Coefficients -- Table \ref{tb:pearson roof} -- between the monthly net heat balance and the monthly EUI are obtained as 0.920 (TMYx), 0.964 (2050), and 0.963 (2080), demonstrating a strong correlation between the metrics. 

\begin{figure}[H]
    \centering
    \includegraphics[width=1\textwidth]{plenum curitiba}
    \caption{Monthly net heat balance across the roof façade in Rio de Janeiro, RJ (ZB8/1A)}
    \label{fig:net heat roof rio}
\end{figure}

The PCM leads to significant heat gains in simulations using future climates files when compared to those exhibited by the insulated roof, nevertheless, it is capable of generating negative monthly net heat balances in the winter and spring months, coinciding with periods with minimum EUI for the envelope. Thermal insulation is capable of significantly reducing heat flow in all climates, resulting in less expressive heat gains than other applications. 

Adding to the north and south façades' analyses, one can observe higher plenum operating temperatures in the case of the roof with PCM in comparison to the structure with resistive thermal insulation in climate projections. The weekly average operating temperature exceeds it by \SI{2}{\celsius} in 2050 and \SI{2.4}{\celsius} in the 2080's climate. Furthermore, the range of this temperature is also significant, contrasting with the other façades. Regardless of the simulated climate, the insulated roof can maintain low levels of heat flow, which guarantees adequate temperatures in the space, reaching a maximum weekly average in the future climate of 2080, with \SI{26.3}{\celsius}. Besides, the various heat flux and temperature peaks, exposed in Figure \ref{fig:temps and hflux roof rio} (a) for the uninsulated roof, can be justified by the temperature balance between the inner face and the surface's core. For the typical meteorological year, this roof produces an average operating temperature of \SI{28.7}{\celsius}, about \SI{3}{\celsius} above the others.

\begin{figure}[H]
    \centering
    \includegraphics[width=1\textwidth]{tq plenum curitiba}
    \caption{Hourly curves of Mean Radiant Temperature in the plenum and Heat Flow by conduction on the internal face of the roof façade during the typical winter week in Curitiba, PR (ZB1/3A).}
    \label{fig:temps and hflux roof rio}
\end{figure}

Although there are \SI{14}{\centi\metre} of PCM on the roof, the material remains liquid during typical weeks in future climates, as expressed by the curves in Figure \ref{fig:pcm roof rio} (b) and (c), increasing thermal transmittance. The plenum's inner roof face and, consequently, the return air are kept at temperatures less advantageous for conditioning, which demand more from the HVAC system to remove thermal energy from the mixture of air blown into the thermal zones.

\begin{figure}[H]
    \centering
    \includegraphics[width=1\textwidth]{pcm plenum curitiba}
    \caption{Hourly behavior of the PCM's physical state applied to the roof surface during a typical winter week in Curitiba, PR (ZB1/3A).}
    \label{fig:pcm roof rio}
\end{figure}

\section{Conclusion} 
The present work proposed to evaluate the thermal and energy behavior in future climates of a commercial building due to the implementation of different materials in the envelope and their impacts on the climate adaptability and thermal resilience of this building. Using the software EnergyPlus \cite{energyplus2019} and Python codes for analysis and output data processing, three types of the envelope were studied: with a resistive thermal insulator, according to ASHRAE, with the phase change material SP24E \cite{sp24e}, geometrically replacing the same insulator and without any insulating material, having the highest thermal transmittance among the considered envelopes. A prototype building energy model was used, made available by the United States Department of Energy \cite{buildingprot}, and whose systems are in accordance with ASHRAE standards. For simulations under different climatic behaviors, two representative cities from bioclimatic zones defined by ABNT criteria (1 and 8) were selected, matching those of the American Association (3A and 1A). 

Projections of future climate change scenarios -- 30 years represented by the years 2050 and 2080 -- were incorporated into the climate files necessary for EnergyPlus simulations through the statistical downscaling method called \textit{morphing}, using the CCWorldWeatherGen tool \cite{ccworldweathergen} and data from the HadCM3 Global Atmospheric Circulation Model, made available by the Intergovernmental Panel on Climate Change (IPCC), for the A2, medium-high emissions scenario \cite{ipcc}. The building-wise metrics consisted of the monthly Energy Use Intensity with HVAC systems and the monthly behavior of the net heat balances through selected envelopes' surfaces. At the thermal zone scale, hourly profiles were studied in typical weeks of the mean radiant temperature, operating temperature and heat flux on the internal face of the façades' surfaces, and also the hourly characteristic physical state of the PCM on the respective surfaces where the material was applied. 

According to the results and discussions, the implementation of the phase change material with operating temperatures within the setpoint range of the air-conditioning system results in better thermal and energy performance of the building in regions with mild climates compared to the use of resistive thermal insulators, presenting lower EUI values. However, in places where climate projections indicate a significant increase in temperatures, the application of PCM may be unfeasible because heat transfer through the surface intensifies while the material remains in a liquid state during the building's operation. In this scenario, insulating materials prove to be more beneficial from the point of view of energy efficiency than PCM. The simultaneous combination of passive applications (in the thermal zone envelope) and active (in the plenum envelope) enhances the impact of implementing the phase change material throughout the external envelope, maintaining adequate levels of mean radiant temperature and operating temperature depending on the HVAC system setpoints.

Besides, climate change appears to be an essential aspect for the evaluation and design of envelopes, and its consideration in energy simulations is a relevant tool for studying the performance of the building during its life cycle. The Pearson Correlation Coefficients calculated make it clear that, with increasing temperatures, heat gains through the envelope, particularly in a building with poor thermal insulation, will dictate energy consumption with air-conditioning, which denotes the imperative of implementing envelope strategies to ensure adaptability and climate resilience of buildings. From this perspective, being a naturally passive and climate-sensitive application, the definition of the PCM operating temperature range must be designed primarily by observing local meteorological conditions, concomitantly with the definition of thermodynamic parameters.

Furthermore, the increase in the thermal conductivity of the phase change material, although it brings benefits from the point of view of the thermal hysteresis cycle, can negatively impact the performance of the envelope where the material is applied since the PCM can remain in a single physical state, thus losing the latent heat storage capacity and increasing the thermal transmittance of the surface where it is applied, intensifying heat gains.

For future research, it is suggested to include the economic feasibility of the envelopes, consideration of different climatic zones, and the application of conventional materials from Brazil's national civil construction as a reference, which already have relevant thermal mass. PCM works based on hysteresis cycles.Depending on the number of cycles, the material may present variations in its thermal properties. In this sense, the periodic replacement of the phase change material plays a relevant role for adapting the envelope transmittance to new weathers or combining the materials into a composite surface. Other suggestions involve the consideration of different HVAC systems to study the passive and active effects of PCM implementation and the impact on partial load performance of air-conditioning equipment. Furthermore, genetic algorithms can parameterize the envelope characteristics in subsequent work to optimize energy efficiency, thermal comfort, and construction and operating costs of the building.

%bibliographystyle{unsrt} 
%bibliographystyle{elsarticle-harv}
%bibliographystyle{elsarticle-num-names}
\bibliographystyle{elsarticle-num}

\bibliography{refs}

\end{document}

